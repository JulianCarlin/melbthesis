Here and henceforth ``the author'' refers to the author of this thesis, Julian Brian Carlin. This thesis is an original work by the author reporting research done alone or in collaboration with other authors. 
This section provides a chapter-by-chapter summary of the author's contributions and the publication status of all material.

\begin{description}
  \item[Chapter 1] is a comprehensive literature review for the work in Chapters 2--6 written by the author for this thesis. It is an original work of the author and will not be submitted for publication.
  \item[Chapter 2] is published as \citet{Carlin2019ac} in the Monthly Notices of the Royal Astronomical Society. This work was written primarily by the author, with scientific input and editing from A.~Melatos. Some ideas in this chapter originally appeared in the thesis submitted for the degree of Master of Science (Physics) awarded to the author in 2018. Figure \ref{fig:acorr_sp} originally appeared as figure 4.5 in the above-mentioned thesis. All figures and tables are the work of the author.
  \item[Chapter 3] is published as \citet{Carlin2020bsa} in the Monthly Notices of the Royal Astronomical Society. It was written primarily by the author, with scientific input and editing from A.~Melatos. All figures and tables are the work of the author.
  \item[Chapter 4] is published as \citet{Carlin2021endog} in The Astrophysical Journal. It was written primarily by the author, with scientific input and editing from A.~Melatos. All figures are the work of the author.
  \item[Chapter 5] is submitted for publication in The Astrophysical Journal. It was written primarily by the author, with scientific input and editing from A.~Melatos and M.~Wheatland. Appendix \ref{app:sf_toy} was written primarily by A.~Melatos, but is included in this thesis for completeness. All figures and tables are the work of the author. 
  \item[Chapter 6] is published as \citet{o3amxp} in Physical Review D on behalf of the LIGO-Virgo-KAGRA (LVK) collaboration. It was written primarily by the author, with scientific input and editing from A.~Melatos, and other members of the LVK collaboration. According to LVK policies, all members of the collaboration are listed as authors on the publication, in alphabetical order. The author was responsible for selecting the targets for the search, developing the scripts to run the search pipeline (which was first developed by \citet{Suvorova2016, Suvorova2017}), running the search, constructing and testing validation procedures for outliers, and calculating upper limits on detectable strain. The analysis was reviewed internally to the collaboration by Pat Meyers and Evan Goetz. The followup search described in the final two paragraphs of Appendix~\ref{app:amxp_shortsax_followup} was performed by Rodrigo Tenorio and David Keitel using the {\sc{PyFstat}} pipeline, thus those paragraphs were written by them, not the author. All figures and tables are the work of the author.
  \item[Chapter 7] summarizes the work in Chapters 2--6. It includes some exploratory future directions for the work in this thesis. It was written by the author, with editing from A.~Melatos. All figures and tables are the work of the author.
\end{description}

During the author's PhD they also contributed to two other publications, which are not included in this thesis but are listed here for completeness.
\begin{itemize}
  \item \citet{Millhouse2022} studies how observed glitch rates scale with the spin frequency, spin frequency derivative (and combinations thereof). The author contributed edits to the manuscript, and scientific input regarding treatment of the likelihood function and related statistics.
  \item \citet{Jones2022} presents a guide for two follow-up procedures for continuous gravitational wave candidates. The author contributed edits to the manuscript, and scientific input regarding the effective point spread function of the matched filtering statistic.
\end{itemize}