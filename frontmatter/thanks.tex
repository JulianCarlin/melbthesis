It has been a long time coming. Over the last four-and-a-bit years I was at times excited, challenged, joyful, depressed, relieved, and more. With the support of many, I am at the finish line. My supervisor Andrew Melatos was correct in warning me when I started that it would not be all sunshine and roses. The ``second-year blues'' coincided neatly with the arrival of COVID-19, and the necessary mental, physical, and social readjustments to work out how to work from home in a vaguely productive manner. If I at all succeeded, it was due to the support of everyone below.

Foremost, I must thank Andrew. His curiosity, intellect, and broad interests have steered me and the projects I have worked on in many varied directions. Despite mostly finding results that seemed like dead-ends, there was always a nugget of knowledge or understanding to explore. His rigour and candour (some might un-generously add persnicketiness) has inspired me to see physics, models, and the world in a new light.

I am also grateful to my supervisory panel, Rob Evans, Andy Martin, and David Simpson. Their excellent advice helped me reign in some of the potential excesses of this thesis. I am also thankful for the research opportunities and mentorship that Rachel Webster and Michele Trenti offered me before I had even started my Masters degree. Their training allowed me to hit the ground running.

Thanks also to the Viterbi, NS/GW, and Astro groups at the University of Melbourne. It was missing the friendly and stimulating environment of the David Caro building's third floor that made the shunting to working from home over 2020 and 2021 so hard. 

Finally, I need to thank my family, friends, and partner for keeping me sane (mostly), work-life-balanced (sometimes), and not-\emph{too}-distracted. 