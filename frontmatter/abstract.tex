Building a model is the only way to search for a signal in noisy data. Many systems governed by stress accumulating steadily and releasing abruptly are difficult to model from the microphysical interactions all the way up to the macroscopic observables. Instead, one can build phenomenological models which encompass the underlying mechanics, and falsify them with data. When the signal is weak compared to the noise, signal models must balance the flexibility required to encompass the stochastic generative process, while maintaining the specificity required to find particular markers of the physics in the data. This thesis explores these ideas in three different contexts. 

Pulsar glitches are sudden, unpredictable jumps in the spin frequencies of some pulsars. The state-dependent Poisson (SDP) process is a framework which models the globally averaged stress in a system as a function of time. The stress accumulates deterministically between events. The instantaneous rate of a stress-release event is a monotonic function of the stress, i.e.~as the stress increases, an event becomes more likely. Once a stress-release event is triggered, some fraction of the stress is released instantaneously. Previous work has shown that for individual glitching pulsars the observed distributions of waiting times between glitches, and the distributions of their sizes, are consistent with the SDP process. The cross-correlation between sizes and the subsequent (or preceding) waiting time are also consistent with the process, and falsifiable predictions are made regarding which pulsars may exhibit such a cross-correlation in the future, as more glitches are discovered. Considering the autocorrelation between consecutive waiting times, or sizes, provides an additional constraint. Even when combining all the above statistical measures, there exists a configuration of the SDP framework which adequately describes the observed sequence of glitch waiting times and sizes, for pulsars with more than 15 recorded glitches. However, as this configuration must vary pulsar-to-pulsar, there is tentative evidence that the underlying mechanism triggering glitches may also vary pulsar-to-pulsar.

If the stress instead accumulates between glitches via a random walk, until a stress threshold is reached, the statistical predictions regarding waiting times and sizes are less permissive than within the SDP framework. This alternative stress-accumulation and relax ``meta-model'' is motivated by glitch trigger mechanisms involving hydrodynamic instabilities, as well as pulse-to-pulse observations of the Vela pulsar showing evidence for a negative fluctuation in spin frequency immediately prior to its 2016 glitch. One key prediction of the Brownian stress accumulation meta-model is that the cross-correlation between the size and subsequent waiting time should be greater than zero in all pulsars, as well as predicting an excess of short waiting times if the cross-correlation is low. The observed sequence of sizes and waiting times for at least two pulsars can only be explained as arising from this meta-model if many small glitches are missing from glitch catalogs.  

A key phenomenological degree of freedom in the SDP framework is the conditional distribution of stress-release event sizes. This degree of freedom is necessary to encompass the variety of possible glitch trigger mechanisms in the literature. However, if one specializes the meta-model, it is possible to (provisionally) falsify individual microphysical mechanisms, for example the idea that glitches are the result of a coherent stress process which triggers superfluid vortex avalanches. The SDP meta-model is augmented such that the amount of stress released at each event is no longer a random variable, however the probability of a glitch is still a monotonic function of the stress in the system. The amount of stress released is calculated endogenously, by tracking the distribution of pinning strengths of occupied vortex pinning sites. Tracking this distribution over time allows for a long-term memory of the past history of stress-release events. This alternative meta-model predicts distinctive statistical observables. For example, there should be a peak and cut-off in both the waiting time and size distributions, corresponding to events that completely reset the system by unpinning all vortices. We do not see this in any glitching pulsar, although we need to observe more glitches to concretely falsify this mechanism. 

Pulsar glitches are not the only physical system which is governed by a stress accumulation and relaxation process. There is broad agreement that solar flares are a sudden release of magnetic energy from the sun's corona. The energy accumulates via sub-photospheric motion, and a flare is more likely to trigger as the energy density increases. The SDP framework is mapped to the context of solar flares, and the hypothesis that solar flares are triggered when the stress reaches a static-in-time threshold is interrogated. If it were true, one should see a cross-correlation between flare sizes and subsequent waiting times, alongside similarly shaped distributions for flare sizes and waiting times. Across $\sim2\times10^3$ active regions and $\sim5\times10^4$ flares, there is no strong evidence for this association in the \emph{Geostationary Operational Environmental Satellite} (\emph{GOES}) historic soft X-ray flare database. If the database is complete, i.e.~not missing many flares, this implies that perhaps flares are triggered before the stress nears the threshold, perhaps the threshold varies in time, or perhaps the rate at which stress accumulates in the system varies in time.

Detecting continuously-emitted quasi-monochromatic gravitational waves is a key goal of the Laser Interferometer Gravitational-Wave Observatory (LIGO), Virgo, and KAGRA collaborations. One such search from accreting millisecond X-ray pulsars is performed using data from the latest LIGO observing run. These targets are promising due to accretion possibly building surface asymmetries (``mountains'') or exciting $r$-mode superfluid oscillations in the neutron star interior. However, direct integration of the data over long time spans is hindered by the varying accretion torque, which causes the spin frequency of the star to wander stochastically. The signal model, in the source frame of the target, is that of a piecewise-constant frequency, which is allowed to randomly wander up to $\sim5\times10^{-7}\,$Hz every ten days. This is implemented via the $\mj$-statistic, which performs the coherent matched filter over ten-day chunks. These statistics are combined with a hidden Markov model to stitch together the most-likely path of the signal, given the data. However, the loudest candidates from the search are consistent with arising from noise. Upper limits are placed on the detectable wave strain amplitude at 95\% confidence, and thus the neutron star ellipticity and $r$-mode amplitude. The strictest of these constraints are from IGR J00291+5934, and are $\epsilon^{95\%} = 3.1\times10^{-7}$ and $\alpha^{95\%} = 1.8\times 10^{-5}$ respectively.
